\documentclass[8pt,oneside]{extbook}
% I would be needing to set a extra features provided by this class as well as I would be needing to setup for smallest font. Smallest font leads to most content

\usepackage{dependencies}
\usepackage{titlepage}
\usepackage{parttoc}
\usepackage{chaptertoc}
\usepackage{questions}
\usepackage{general}

\usepackage{fontspec}
\setmainfont{Lexend}

\geometry{
  left=0.5cm,
  right=0.5cm,
  top=0.5cm,
  bottom=0.5cm,
}


% \texorpdfstring{}
% texorpdfstring{}
% IfvalueT/F
% \NewDocumentCommand \RenewDocumentCommand \DeclareDocumentCommand
% \NewDocumentCommand{\foo}{O{bar}mmmm}{ ... }
% \let -> \NewCommandCopy
% \defcounter \deflength
% \enquote
% \overset
% \vspace \hspace \mspace 
% \ForEach \ForEachSublevel
% \FunctionForEach
% \ifnumcomp \ifdimcomp

% Number of parts are limited to 16

\setcvFont{42}{18}{16}





\registerchapter{ch-hollow}{chaptertitle}
\registersection{sec-absence}{ch-hollow}
\registersection{sec-carto}{ch-hollow}

\registerchapter{ch-another}{chaptertitle}
\registersection{sec-only}{ch-another}

\registerpart{chaptertitle}{On the Nature of Nothing}



\begin{document}
	\pagenumbering{roman}
    \makecoverpage
    \makeparttoc

	\pagenumbering{arabic}
	
    \createpart{chaptertitle}{On the Nature of Nothing}
    \makechaptertoc{chaptertitle}


    \createchapter{ch-hollow}{The Hollow at the Centre}
    \createsection{sec-absence}{What Absence Tastes Like}
    \createsection{sec-carto}{A Cartography of Gaps}

    \createchapter{ch-another}{Another One}
    \createsection{sec-only}{Its Only Section}



    \begin{multicols}{2}
	
	\begin{Boxy}
		Here is something here
	\end{Boxy}
	
	\begin{DefBox}{Some term}
		Some Definition
	\end{DefBox}
	
	\end{multicols}
	
	
	
	\begin{Boxy}
		Page wide text 
	\end{Boxy}
	
	\begin{Proof}
		\begin{ClaimBox}
			Claim: I am not alive
		\end{ClaimBox}
		\begin{ProofSteps}
			Is this working
		\end{ProofSteps}
		\begin{ProofSteps}
			Is this working
		\end{ProofSteps}
		\begin{ProofSteps}
			Is this working
		\end{ProofSteps}
		\begin{ProofSteps}
			Is this working
		\end{ProofSteps}
		Thanks and regards'
		
		
		\ProofStep{
			Hi
		
		}
	\end{Proof}

	\begin{codebox}{c}{Binary Search Implementation}
		int main() {
			// Print hello world
			printf("Hello, World!\n");
			return 0;
		}
	\end{codebox}
	
	
	
	\begin{codebox}{cpp}{Something}
#include <vector>
		
int binarySearch(std::vector<int>& arr, int target) {
	int left = 0;
	int right = arr.size() - 1;
				
	while (left <= right) {
		int mid = left + (right - left) / 2;
					
		if (arr[mid] == target)
		return mid;  // Found it!
		else if (arr[mid] < target)
		left = mid + 1;
		else
			right = mid - 1;
		}
return -1;  // Not found
}
	\end{codebox}

	
	\begin{multicols}{2}

	\section{Implementation}
	
	Here's the implementation of our sorting algorithm:

	\begin{codebox}{Bash}{Another ease tester}
		void bubbleSort(int arr[], int n) {
			// Outer loop for passes
			for (int i = 0; i < n-1; i++) {
				// Inner loop for comparisons
				for (int j = 0; j < n-i-1; j++) {
					if (arr[j] > arr[j+1]) {
						// Swap elements
						int temp = arr[j];
						arr[j] = arr[j+1];
						arr[j+1] = temp;
					}
				}
			}
		}
	\end{codebox}

	The algorithm has a time complexity of $O(n^2)$.
	
	
	
	\end{multicols}
	
	
	\begin{example}{blue}
		Hi
	\end{example}
	
	\begin{solution}{green}
		This is solution
	\end{solution}
	
	
	\begin{example}
		Solve for $x$: $2x + 5 = 13$
	\end{example}
	
	\begin{solution}
		Subtract 5 from both sides: $2x = 8$
	\end{solution}
	
	\begin{solution}
		Alternative method: We can also write it as $x = \frac{13-5}{2} = 4$
	\end{solution}
	

	
	\begin{example}
		Find the derivative of $f(x) = x^2 + 3x$
	\end{example}
	
	\begin{solution}
		Using power rule: $f'(x) = 2x + 3$
	\end{solution}
	
	
	
	\section{Change of Section}
	
	
	\begin{example}{blue}
		Hi
	\end{example}
	
	\begin{solution}{green}
		This is solution
	\end{solution}
	
	
	\begin{example}
		Solve for $x$: $2x + 5 = 13$
	\end{example}
	
	\begin{solution}
		Subtract 5 from both sides: $2x = 8$
	\end{solution}
	
	\begin{solution}
		Alternative method: We can also write it as $x = \frac{13-5}{2} = 4$
	\end{solution}
	
	
	
	\begin{example}
		Find the derivative of $f(x) = x^2 + 3x$
	\end{example}
	
	\begin{solution}
		Using power rule: $f'(x) = 2x + 3$
	\end{solution}
	
	
	\begin{multicols}{2}
		
		\begin{example}{blue}
			Hi
		\end{example}
		
		\begin{solution}{green}
			This is solution
		\end{solution}
		
		
		\begin{example}
			Solve for $x$: $2x + 5 = 13$
		\end{example}
		
		\begin{solution}
			Subtract 5 from both sides: $2x = 8$
		\end{solution}
		
		\begin{solution}
			Alternative method: We can also write it as $x = \frac{13-5}{2} = 4$
		\end{solution}
		
		
		
		\begin{example}
			Find the derivative of $f(x) = x^2 + 3x$
		\end{example}
		
		\begin{solution}
			Using power rule: $f'(x) = 2x + 3$
		\end{solution}
		
		
		
	\end{multicols}
	
	\begin{codeexample}
		Write a function to swap two integers:
		
		\begin{codebox}{c}{test}
			void swap(int *a, int *b) {
				int temp = *a;
				*a = *b;
				*b = temp;
			}
		\end{codebox}
	\end{codeexample}
	
	
	
	\begin{codesolution}
		\begin{codebox}{c}{Test example}
			// Solution code here
		\end{codebox}
	\end{codesolution}
	
	
	
	\begin{myTheorem}{Fermats Last Theorem}{thm:FermatsLastTheorem}
		No three positive integers \(a\), \(b\) and \(c\) satisfy the equation \(a^{n} + b^{n} = c^{n}\) for any integer greater than two.
	\end{myTheorem}
	
	
	
	
	Here is a simple comment:
	\pdfcomment{Just testing a basic note.}
	
	\bigskip
	Here is a highlighted section:
	\pdfmarkupcomment[markup=Highlight, color=yellow]
	{important concept}{Explain this more clearly.}
	
	\bigskip
	Here’s a strikeout:
	\pdfmarkupcomment[markup=StrikeOut, color=red]
	{remove this}{This is incorrect.}
	
	\bigskip
	Underline example:
	\pdfmarkupcomment[markup=Underline, color=blue]
	{keep this}{This is correct and must remain.}
	
	\bigskip
	Tooltip example:
	\pdftooltip{Hover over this word}{This is hidden info!}
	

	










	\begin{interactivequestion}{Fibonacci Sequence}
		% =========================
		% TOP ROW
		% =========================
		\noindent
		\begin{minipage}[t]{0.24\textwidth}
		\vspace{0pt}
		\begin{ocg}{KeyPoints}{keylayer}{0}
		\begin{cornerbox}{blue}{Key Points / Taxonomy}
		Bloom: Apply  

		SOLO: Relational  

		Recursion  

		Dynamic Programming
		\end{cornerbox}
		\end{ocg}
		\end{minipage}
		\hfill
		\begin{minipage}[t]{0.48\textwidth}
		\vspace{0pt}

		\begin{questionbox}{Question}
		Write a function for the nth Fibonacci number.
		\end{questionbox}

		\vspace{6mm}

		\centering
		\ocgbutton{keylayer}{Key Points}
		\ocgbutton{variationlayer}{Variations}
		\ocgbutton{reslayer}{Resources}
		\ocgbutton{hintlayer}{Hint}
		\ocgbutton{sollayer}{Solution}
		\ocgbutton{anslayer}{Answer}

		\end{minipage}
		\hfill
		\begin{minipage}[t]{0.24\textwidth}
		\vspace{0pt}
		\begin{ocg}{Resources}{reslayer}{0}
		\begin{cornerbox}{orange}{Resources / Expected Time}
		Expected Time: 5 minutes  

		Pen and Paper allowed  

		No calculator
		\end{cornerbox}
		\end{ocg}
		\end{minipage}

		% =========================
		% SPACE BETWEEN ROWS
		% =========================
		\vspace{10mm}

		% =========================
		% SECOND ROW
		% =========================
		\noindent
		\begin{minipage}[t]{0.24\textwidth}
		\vspace{0pt}
		\begin{ocg}{Variations}{variationlayer}{0}
		\begin{cornerbox}{purple}{Possible Variations}
		Iterative version  

		Matrix exponentiation  

		Memoized recursion  

		Tail recursion
		\end{cornerbox}
		\end{ocg}
		\end{minipage}
		\hfill
		\begin{minipage}[t]{0.48\textwidth}
		\vspace{0pt}

		\begin{ocg}{Hint}{hintlayer}{0}
		\begin{hintbox}
		Base case: if n is less than or equal to one, return n.  

		The recurrence relation is F(n) equals F(n-1) plus F(n-2).
		\end{hintbox}
		\end{ocg}

		\vspace{10mm}

		\begin{ocg}{Answer}{anslayer}{0}
		\begin{onewordbox}
		\Huge Recursion
		\end{onewordbox}
		\end{ocg}

		\end{minipage}
		\hfill
		\begin{minipage}[t]{0.24\textwidth}
		\vspace{0pt}
		\begin{ocg}{SolutionLink}{sollayer}{0}
		\begin{cornerbox}{green}{Solution Link}
		See detailed solution on page \pageref{fibo_solution}
		\end{cornerbox}
		\end{ocg}
		\end{minipage}

	\end{interactivequestion}













	\newpage
	Hello

	\newpage
	Another hi

	\newpage
	\phantomsection
	Hi \label{fibo_solution}


\end{document}
